%!TEX root=../main.tex

\subsection{Beschreibung Denkweise}
\subsubsection{\textbf{Design Thinking}}
Design Thinking und Human Centered\cite{slidesDesign} beschreibt den Design Prozess und wie das Design auf Nutzer und Nutzerinnen wirkt. Im Falle der Smart Devices beispielsweise geht es darum die Geräte so zu gestalten, dass zusätzliche ''smarte'' Funktionen bestehen bleiben, ohne das die Hauptfunktionalität verloren geht/darunter leidet. Bei der Gabel zum Beispiel soll der Nutzer/die Nutzerin damit noch angenehm essen können. \\
Das Hauptproblem liegt darin die Hauptfunktionalität beizubehalten, zusätzliche Funktionen einzubauen und dem Nutzer/der Nutzerin Feedback zu geben, ohne das dieser/diese sich genervt oder genötigt fühlen. Im Falle von Thomas\cite{uninvatedGuests} ist dies nicht ganz der Fall, da er beim nicht Benutzen der Gegenstände permanent Benachrichtigungen und Erinnerungen bekommt.
\subsubsection{\textbf{Creative Thinking}}
Creative Thinking\cite{slidesCreative} handelt von Kreativität und dem kreativen Prozess bei beispielsweise dem Design eines Gegenstandes. Vor allem bei den im Video\cite{uninvatedGuests} gezeigten Smart Devices ist eine kreative Lösung erforderlich um die Hauptfunktionalität eines Alltagsgegenstandes mit ''smarten'' Funktionen zu erweitern. Beispielsweise bei der gezeigten Gabel\cite{uninvatedGuests} musste Platz geschaffen werden um die Elektronik einzubauen. Hierbei wurde kreativ gehandelt und der Griff der Gabel wurde breiter gestaltet um Platz für die Elektronik einzubauen.
\subsection{Analyse des Videos}
\subsubsection{\textbf{Generelle Dokumentation zum Design}} 
Bei dem Design der verschiedenen Smart Devices wurde hauptsächlich das Ziel verfolgt die Geräte nicht zu sehr zu verändern. Die Geräte sollen in der Bedienung gleich wie die Vorgängerobjekte sein. Das Ziel ist es ja nicht die Nutzung zu optimieren, sondern durch gesammelte Daten eine Person gesünder zu machen oder diese Daten zu merken. \\
Die Smart Devices sind im Video hervorgehoben, da diese die Hauptrolle im Video darstellen. Die gewählte Farbe Gelb hebt die Gegenstände deutlich hervor und die Gegenstände wirken abstrakt verglichen mit den anderen Gegenständen. Zudem sind manche Smart Devices abstrakter im Aufbau gestaltet. Die Gabel ist beispielsweise dicker als eine normale Gabel, dies wurde wahrscheinlich aufgrund der vorhanden und eingebauten Elektronik gemacht. \\
\subsubsection{\textbf{Tag 1}} 
Am ersten Tag von Thomas alltäglichen Leben wird die Funktionsweise der Smart Devices vorgestellt. \\
Die Gabel soll Thomas dabei helfen seine Ernährung gesünder zu gestalten, indem das gegessene Essen mittels Sensoren überprüft wird. \\
Der Gehstock ist eine Art Schriftzähler, welcher überprüft, ob Thomas auch genug Bewegung pro Tag macht. \\
Die Matratze erinnert Thomas, wann er am besten Schlafen gehen soll. \\
Alle Smart Devices sind mit dem Handy verbunden um dem Nutzer/der Nutzerin Benachrichtigungen über den aktuellen Stand zu schicken, beziehungsweise Erinnerungen, falls Thomas etwas vergisst.
\subsubsection{\textbf{Tag 2}} 
Thomas ignoriert die Smart Devices und benutzt diese nicht. Zuerst wird Thomas mit Erinnerungen von den Smart Devices zugeschüttet, dass er die Smart Devices benutzen soll. Diese Erinnerungen ignoriert er auch, jedoch benachrichtigen die Smart Devices auch seine Kinder und diese melden sich dann. Thomas Kinder überwachen Thomas also und informieren sich über seine Gesundheit mittels den Smart Devices.
\subsubsection{\textbf{Tag 3}} 
Thomas trickst die Smart Devices aus und verschafft sich so Ruhe von den Erinnerungen und auch seinen Kindern, welche Ihn überwachen. Auf den ersten Blick wirkt es als würde Thomas seine Gesundheit (Er kann problemlos ungesund essen, kann den ganzen Tag lesen/fernsehen und hat einen unregulierten Schlafrhythmus) negativ beeinflussen. Jedoch hat das Austricksen auch positive Seiten. Er kann wie gewohnt sein Leben leben, ohne das er von Smart Devices und seinen Kindern überwacht wird und seine Kinder machen sich weniger Sorgen um ihn, da sie sehen, dass alles passt. \\
Generell ist der Designvorschlag mit der Überwachung nicht gut. Ein Mensch sollte sein Leben frei leben können. Vor allem das die Kinder durch gewisse Apps seine Gesundheit überwachen und nicht persönlich Vorbeischauen finde ich ein wenig unmenschlich. Wenn sie sich wirklich Sorgen um Thomas machen, dann könnten sie ihn auch von Zeit zu Zeit besuchen. \\
\subsection{Problem Framing}
Das Smart Devices wird nicht direkt für den tatsächlichen Nutzer/der tatsächlichen Nutzerin designt, sondern für die Personen, welche es kaufen für die Überwachung vom gesundheitlichen Zustand des tatsächlichen Nutzers/der tatsächlichen Nutzerin. In Thomas Fall wurde das Smart Device also nicht für ihn, sondern für seine Kinder designt, welche ihn überwachen und das ist auch schon das grundlegende Problem. Thomas ist der Nutzer von den Objekten und er sollte diese auch selbst steuern können. Es wird zwar nicht erwähnt, wer Zugriff auf die Einstellungen (z.B. Bettzeit) von den Smart Devices hat, jedoch denke ich das die Kinder die Administrationsrechte haben, da sie auch die Daten einsehen können. Somit hat Thomas weniger Freiheit über sein Leben.\\
Ein zusätzliches Problem, welches ich bemerkt habe, ist, dass die Person keine Kurzzeitmotivation hat. Zwar ist die Langzeitmotivation ein gesundes Leben führen zu können, jedoch gibt es keine Kurzzeitmotivation und die Benachrichtigungen, das der Nutzer/die Nutzerin seine Ziele erreicht haben, könnten für die Person nervig werden. \\
Das abstrakte Design könnte einige Personen vor der Nutzung abschrecken und wahrscheinlich werden einige Menschen sehr skeptisch gegenüber der Datensammelung\cite{euroSkeptisch} eingestellt sein.\\
Zudem könnten einige Menschen generell Angst vor Alltagsgegenständen bekommen, wenn die Smart Devices sich nicht mehr von den Alltagsgegenständen unterscheiden werden können. Das hätte zur Folge, dass Personen anderen Personen die Smart Devices unterjubeln können um diese ohne ihr Wissen zu überwachen.
\subsection{Gestaltungsvorschlag}
An sich sind die Smart Devices nicht schlecht designt und auch von ihrer Funktion nützlich. Jedoch finde ich sollte Thomas die vollständigen Administrationsrechte für die Smart Devices bekommen und er sollte auch entscheiden dürfen, ob seine Kinder Einsicht auf die gesammelten Daten von den Geräten bekommen sollten. \\
Außerdem könnte der Gestaltungsprozess mittels Prototypen verbessert werden. Die Entwickler und Entwicklerinnen könnten somit für verschiedene Personengruppen die Gegenstände bauen und mittels Prototypen testen, ob das Smart Device für diese zugeschnitten ist. Vor allem sollte dem Nutzer/der Nutzerin Rechte gegeben werden die Ziele selbst bestimmen zu können.\\
Zudem sollten die Smart Devices eine Art Pause Modus haben, in welchem Thomas nicht unbedingt auf alles achten muss ohne gleich mit Benachrichtigungen voll geschüttet zu werden. Selten etwas ungesundes zu Essen, einen faulen Couchtag zu haben und auch die regulierte Schlafzeit zu ignorieren wirkt sich nicht negativ auf die Gesundheit aus. \\
Das Problem mit der Kurzzeitmotivation könnte so gelöst werden, dass die Person Punkte und Ränge sammelt, wenn die Person die Ziele erreicht. Somit entsteht ein virtuelle Glückserfahrung und die Person wird auch auf lange und kurze Zeit motiviert weiterzumachen. So ein System wurde beispielsweise bei dem Spiel Habitica\cite{habitica} eingesetzt. \\
Ich finde Smart Devices sollten außerdem in irgendeiner Art gekennzeichnet sein. Denn falls dies nicht der Fall wäre, könnten Personen die Smart Devices einer anderen Person unterjubeln um unwissentlich Daten zu sammeln. Im gezeigten Video\cite{uninvatedGuests} sind beispielsweise die Smart Devices sehr auffällig markiert, jedoch sollte auch ein einfaches Siegel auf dem Gegenstand reichen um diesen von Alltagsgegenständen unterscheidbar zu machen.
