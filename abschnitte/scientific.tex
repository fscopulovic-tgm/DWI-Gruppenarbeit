%!TEX root=../main.tex

\subsection{Was macht Scientific Thinking aus?}
Scientific Thinking beschäftigt sich mit dem Versuch eindeutige, gut begründete, geordnete und für gesichert erachtete Ergebnisse hervorzubringen. Das Ganze verfolgt das Ziel „unbestreitbares“ Wissen zu schaffen und darauf weiter aufbauen zu können. \\
Der Prozess dahinter läuft so ab, dass zuerst bestimmte Abläufe/Sachverhalte genau beobachtet werden, anschließend wird anhand dieser Beobachtungen eine Hypothese aufgestellt welche versucht die Beobachtungen zu erklären bzw. das beobachtete Problem zu lösen. Nachdem eine Hypothese aufgestellt wurde muss versucht werden diese oft genug zu überprüfen damit eindeutig gezeigt wird das die Hypothese allgemein gültig ist und nicht nur in einem überprüften Einzelfall gilt. Je nachdem ob die Hypothese der Überprüfung stand hält wird sie danach entweder als hervorgebrachtes Wissen betrachtet, immer weiter verbessert bis sie wissen hervorbringt oder wenn die Ergebnisse zu weit von den erwarteten entfernt ist auch einfach wieder verworfen.

\subsection{Analyse}
Grundlegend baut das System \cite{uninvatedGuests} schon mal auf Scientific Thinking auf. Das System kann nur dadurch funktionieren, dass irgendwann wissenschaftlich herausgefunden wurde wie lange Menschen täglich schlafen sollten, wie viel sie wovon essen sollten und wie viele Schritte man am Tag zumindest gehen sollte. Die Daten die erhoben werden müssen um zu überprüfen ob die „Anforderungen“ eingehalten werden entstehen bei der „Smart-Gabel“ durch Sensoren die den Nährwert der Speisen messen wenn man sie aufspießt, bei dem Bett durch Gewichtssensoren die Messen ob jemand/etwas im Bett liegt und beim „Smart-Gehstock“ wird durch Bewegungssensoren gemessen wie weit der Stock sich fortbewegt. Diese Daten werden an ein externes System übermittelt welches sie dann mit dem Optimum vergleicht und je nachdem ob dieses erreicht wurde positives oder negatives Feedback gibt. \\
Auf der anderen Seite könnte man auch das Verhalten des Mannes als Scientific Thinking betrachten.  \\
Am ersten Tag beobachtet er wie die Geräte darauf reagieren, wenn er sich verhält wie immer. Er lässt sich zwar zu ein paar extra Schritten motivieren gibt das Ganze aber dann relativ schnell wieder auf. Da das Feedback des Systems am ersten Tag sehr negativ ausfällt stellt er die Hypothese auf, dass es doch wesentlich angenehmer wäre, wenn diese Geräte ihn nicht ständig daran erinnern würden was er Essen sollte, wie viel er sich bewegen sollte und wann er schlafen sollte.  \\
Daraufhin entschließt er sich diese Hypothese am nächsten Tag zu überprüfen indem er die Geräte neben sich liegen lässt beziehungsweise sie so verstaut das diese ihm nicht mehr durch ständige Erinnerungen auf die Nerven geht. Dies gelingt jedoch leider nicht da ihm seine Kinder daraufhin auf die Tatsache ansprechen, dass die Geräte nicht verwendet werden. Das war natürlich nicht sein gewünschtes Ergebnis da er obwohl die Geräte nicht benutzt wurden trotzdem auf sein „Fehlverhalten“ hingewiesen wurde.  \\
Darum entschließt er sich seine Hypothese zu verbessern und die Geräte am dritten Tag nicht mehr nicht zu verwenden, sondern sie auszutricksen. Das macht er indem er mit seiner „Smart-Gabel“ nicht wirklich isst, sondern damit nur in gesundem Essen herumstochert, seinen „Smart-Gehstock“ nicht selbst benutzt, sondern ihn jemanden anderen gibt um die Bewegung für ihn zu machen und indem er die Gewichtssensoren des Bettes hinters Licht führt indem er es mit Büchern beschwert. 
Nachdem das Feedback welches er zu diesem Verhalten erhält ausschließlich positiv ist hat er durch beobachten, anschließende Hypothesen aufstellen/überprüfen und daraus Schlussfolgerungen ziehen das Wissen erlangt wie er mit diesen Geräten leben kann ohne das sie ihm auf die Nerven gehen.  

\subsection{Problem Framing}
Das Hauptproblem des Ansatzes dieser Geräte ist, dass Angenommen wird das Menschen, wenn sie darauf hingewiesen werden das andere Verhaltensmuster wissenschaftlich gesünder für sie wären, sie ihr Verhalten dementsprechend anpassen würden. Jedoch ist es zum einen nun mal so, dass Menschen nicht immer das tun was für ihre physische Gesundheit am besten ist, wenn man dafür seine persönlichen Wünsche vernachlässigen muss. \\
Ein weiteres Problem besteht darin, dass diese Daten nicht zwingender maßen vertrauenswürdig sind, sie aber das einzige Kriterium für den Erfolg der Technologie sind. Wie man bei dem Mann in dem Video schon gesehen hat sind diese relativ leicht zu verfälschen, wenn man sich nicht an die „Anforderungen“ halten möchte und damit ist der Sinn des Systems sehr schnell übergangen und dadurch auch zur Gänze nutzlos. 

\subsection{Gestaltungsvorschlag}
Um diese beiden Probleme beheben zu können müsste man dafür sorgen, dass das System Benutzerfreundlicher wird und dafür gesorgt wird das der Benutzer die gesetzten Ziele auch wirklich selbst einhalten will.  
Dies wäre am einfachsten möglich indem man bei der Einrichtung des Systems genauer auf die Wünsche und Gewohnheiten des Nutzers eingeht. Dafür wäre zum Beispiel erforderlich, dass das System nicht ab der ersten Benutzung „Anforderungen“ stellt die dem Optimum entsprechen, sondern zunächst einmal für einen gewissen Zeitraum (ca. eine Woche) nur misst wie das Ess-, Schlaf- und Bewegungsverhalten der Person normalerweise ist. \\
Anschießend können je nachdem unterschiedliche „Anforderungen“ gestellt werden die sich den persönlichen Gewohnheiten anpassen und sich dann mit der Zeit Schritt für Schritt immer weiter ans Optimum annähert. Dadurch würden wesentlich kleinere Schritte schon ein positives Feedback auslösen und das würde sehr zur Motivation beitragen die Geräte langfristig zu benutzen und sich Schritt für Schritt ans Optimum anzunähern. 