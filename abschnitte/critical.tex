%!TEX root=../main.tex

\subsection{Beschreibung der Denkweise}
\subsubsection{\textbf{Responsible Thinking}}  
Beim Responsible Thinking\cite{slidesResp} ist es wichtig, sich über die Auswirkungen des Systems auf die Gesellschaft und die Umwelt Gedanken zu machen. Durch die Verwendung/Entwicklung von Produkten hat man auch eine gewisse Verantwortung. Das Responsible Thinking geht über die gesetzlichen Anforderungen hinaus und befasst sich auch mit ethischen Fragen, wobei sowohl die eigenen als auch die gesellschaftlichen Werte hinterfragt werden. Aktuelle und wichtige Punkte bei den ethischen Fragen sind die Datensicherheit und die Überwachung. 
\\
\subsubsection{\textbf{Critical Thinking}}  
Beim Critical Thinking\cite{slidesCritical} sollen alle Seiten eines Problems beleuchtet werden, um so Voreingenommenheit zu beseitigen und Stereotypen zu entkräften. Dabei ist es wichtig, auch auf Kritikpunkte einzugehen, die nicht der eigenen Meinung entsprechen. Wenn eine Innovation vom Großteil der Gesellschaft gutgeheißen wird, muss dies nicht bedeuten, dass sie auch für eine Minderheit tragbar ist. Beim vorliegenden Problem sind vor allem die Überwachung und, aufgrund des Systemdesigns, eine mögliche Stigmatisierung kritisch zu betrachten. 

\subsection{Analyse der im Video „dokumentierten“ Nutzung des Systems}
Für das Responsible Thinking ist es wichtig, die ethischen und moralischen Aspekte einer solchen Erfindung zu beleuchten. 
Ist es moralisch korrekt, einen Menschen ständig zu überwachen? Sollte ein Mensch nicht die Freiheit haben, tun und lassen zu können, was er/sie möchte? Sollte man nicht selbst entscheiden können, was man wann isst, welche Hobbies man pflegt bzw. wann man schlafen geht? \\
Eine ständige Überwachung mindert die Lebensqualität des Menschen. Diese Feststellung muss nun mithilfe des Critical Thinking überprüft werden. Aus Sicht der betroffenen Familienmitglieder ist das System sinnvoll und praktisch. Sie haben einen direkten Einblick in den Tagesablauf der „überwachten“ Person. Essverhalten, Bewegung und Schlafenszeit können gemessen und überprüft werden. Dies alles kann aus Sorge um die/den Angehörige(n) und ohne Hintergedanken geschehen, aber auch um das eigene Gewissen zu beruhigen, dass man keine Zeit hat, sich selbst um diese Person zu kümmern. \\
Für die „überwachte“ Person ist diese Überwachung jedoch einengend und in gewisser Weise menschenunwürdig. Sie wird in ihren Freiheiten eingeschränkt und bevormundet. Sie darf nicht selbst entscheiden, was sie isst, wann sie Bewegung machen will oder auch nicht, ja selbst wann sie schlafen gehen soll, wird vorgegeben. Das hier gezeigte System verstärkt noch die vorherrschende Meinung in unserer Gesellschaft, dass ältere Menschen unselbständig sind und Hilfe benötigen. \\
Ein ebenso wichtiger Punkt ist eine durch dieses System möglicherweise auftretende Stigmatisierung älterer Leute. Durch das Design des Systems im Video kann es sein, dass das System von vielen Personen der Zielgruppe abgelehnt wird. Das Design lässt darauf schließen, dass man aus der Norm fällt, also z.B. hilfsbedürftig erscheint. Gabel und Gehstock sind aufgrund des Designs nicht unbedingt für außerhalb der eigenen Wohnung geeignet. Vor allem die Farbe dieser Gegenstände ist sehr auffällig. In der Öffentlichkeit würde man damit Aufsehen erregen bzw. auf jeden Fall auffallen. Man könnte daher z.B. als „alt und gebrechlich“ eingestuft werden. Daher könnte dieses System in der Öffentlichkeit zu einer Stigmatisierung der Zielperson führen.

\subsection{Tagesanalyse}
\subsubsection{\textbf{Tag 1}}
Das smarte System (Gabel, Gehstock, Bett) gibt dem Mann (der Zielperson) die entsprechenden Anweisungen. Die Gabel zeigt an, wieviel Salz, Fett und Kalorien in der Mahlzeit enthalten sind bzw. wie oft am Tag gegessen werden soll. Der Gehstock gibt die Anzahl der Schritte pro Tag vor und das Bett zeigt an, wann es Schlafenszeit ist. Der gesamte Tagesrhythmus wird durch das smarte System vorgegeben. Für selbständige Entscheidungen bzgl. Essen, Spaziergang und Schlafenszeit ist kein Spielraum gegeben.
\subsubsection{\textbf{Tag 2}}
Auch an diesem Tag gibt das smarte System die notwendigen Anweisungen. Diese werden jedoch von der Zielperson ignoriert. Prompt erfolgt eine Rückmeldung der Kinder, ob alles in Ordnung sei bzw. warum das smarte System nicht benutzt wird. \\
Die Kinder sorgen sich auf der einen Seite um das Wohl der Zielperson, daher haben sie ja auch für diese das smarte System besorgt. Auf der anderen Seite beschränkt sich der Kontakt zur Zielperson jedoch nur darauf, zu eruieren, warum das System nicht benutzt wird. Der direkte Kontakt zwischen den Kindern und der Zielperson ist nicht gegeben. Die Kinder telefonieren nicht, sondern schicken eine SMS. Dies kann das Gefühl von Alleinsein noch verstärken, da man keinen Ansprechpartner für seine Probleme hat. Die äußere Kontrolle schränkt die Selbstentscheidung der Zielperson stark ein. Sie fühlt sich bevormundet und nicht als selbstbestimmte Person.
\subsubsection{\textbf{Tag 3}}
Die genervte Zielperson überlistet das System und erntet von diesem „dafür“ Lob. Die Zielperson stochert mit der Gabel im gesunden Essen, lässt jemand anderen die vorgegebene Schrittanzahl tun und gaukelt dem Bett vor, darin zu liegen und zu schlafen. \\
Die Zielperson scheint sich sichtlich wohler zu fühlen. Außer der sehr positiven Rückmeldung des Systems wird sie auch nicht von den Kindern wegen Fehlverhaltens oder Nichtbenutzung des Systems überwacht und „kritisiert“. 

\subsection{Problem Framing}
Im Video\cite{uninvatedGuests} erkennt man deutlich, dass das smarte System starken Einfluss auf den persönlichen Tagesablauf und die Privatsphäre der Person hat. Egal ob beim Essen, beim Lesen oder Fernsehen, immer wird die Person vom System darauf aufmerksam gemacht, dass etwas anderes zu tun sei. Verwendet sie das smarte System nicht, wird sie von den Familienmitgliedern kontaktiert, ob ohnehin alles in Ordnung sei. Die Haltung der Kinder spiegelt die Haltung der Gesellschaft gut wider, dass ältere Menschen nicht für sich selbst sorgen bzw. auf ihre Gesundheit achten können und daher ständig kontrolliert und „bemuttert“ werden müssen.  \\
Ein großes Problem stellen auch die hochgesteckten Ziele des Systems dar. Die Zielperson hat sichtlich Probleme mit den Füßen und daher tut sie sich beim Gehen sichtlich schwer. Daher würde die Person den Tagesablauf viel lieber selbst planen und den Tag gemütlich verbringen. Das smarte System ist in diesem Fall ein Stressfaktor für die Zielperson. \\
Die Nutzung der durch das smarte System gespeicherten Daten stellt ebenso ein großes Problem dar. So könnten diese Daten von der Herstellungsfirma an Dienstleister wie Versicherungen weitergegeben werden. Dieser Datenmissbrauch könnte z.B. zu unterschiedlich hohen Versicherungsprämien führen, was wiederum zu einer Diskriminierung jener Personen führen könnte, die das smarte System nutzen.  

\subsection{Gestaltungsprozess}
Um zu einem besseren Entwurf zu kommen, müsste das smarte System mehr an die jeweilige Zielperson angepasst werden. Eine schrittweise Umstellung von den derzeitigen Lebensgewohnheiten auf die gesündere Lebensweise wäre empfehlenswert. Auch sollte das System an die jeweiligen gesundheitlichen Gegebenheiten der Zielperson adaptiert werden können. \\
Wichtig wäre daher bei der Implementierung eines solchen Systems die Mitwirkung eines Arztes oder einer Ärztin des Vertrauens. Dadurch könnten der Nutzen und die Ziele dieser Systeme besser an die Zielpersonen vermittelt werden und gemeinsam Pläne zur Umsetzung durch schrittweise Anpassung an die Lebensgewohnheiten gemacht werden. \\
Um Missbrauch der Daten und eine Überwachung zu verhindern, sollte es die Möglichkeit geben, diese „Überwachung durch Dritte“ zu deaktivieren. Dies könnte durch Speicherung der Daten am jeweiligen Gerät erfolgen, sodass Dritte darauf keinen Zugriff haben oder nur mit Einverständnis der jeweiligen Zielperson. \\
Als Gesellschaft sind wir gefordert umzudenken. Ältere Menschen mögen gebrechlich sein, dies gibt uns jedoch nicht das Recht, sie zu entmündigen und ihnen vorzuschreiben, was sie zu tun haben. Jede(r), auch ältere Personen, hat das Recht, sein Leben, seinen Tagesablauf so zu gestalten wie sie/er es will. Das Leben lebenswert erhalten, auch im Alter, sollte das Ziel auch von smarten Systemen sein. Wichtig ist die Einbindung solcher System in den Lebensalltag, jedoch sollen solche Systems nicht das Alltagsleben kontrollieren. 