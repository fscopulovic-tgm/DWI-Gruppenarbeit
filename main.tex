\documentclass[sigchi-a, authorversion]{acmart}
\usepackage{booktabs} % For formal tables
\usepackage{ccicons}  % For Creative Commons citation icons
\usepackage[utf8]{inputenc}
\settopmatter{printacmref=false}

% Copyright
\setcopyright{none}


% DOI
\acmDOI{00.000/00.0}

% ISBN
\acmISBN{123-4567-24-567/08/06}

%Conference
\acmConference[DENKWEISEN'18]{Denkweisen der Informatik}{Wintersemester 2018}{Wien, Österreich}
\acmYear{2018}
\copyrightyear{2018}


\begin{document}

\title{Beyond the Uninvited Guest}

\author{Beischlager Georg}
\affiliation{%
  \institution{TU Wien, 01325245}
  }
\email{e1325245@student.tuwien.ac.at}

\author{Heimerl Felix}
\affiliation{%
  \institution{TU Wien, 11728801}
  }
\email{e11728801@student.tuwien.ac.at}

\author{Heissenberger Christopher}
\affiliation{%
  \institution{TU Wien, 01623652}
  }
\email{e1623652@student.tuwien.ac.at}

\author{Scopulovic Filip}
\affiliation{%
  \institution{TU Wien, 11810848}
  }
\email{e11810848@student.tuwien.ac.at}

% The default list of authors is too long for headers.
\renewcommand{\shortauthors}{F. Author et al.}

\maketitle

\begin{abstract}
    Eine Zusammenfassung des Papers (150 Wörter)

\end{abstract}

% =============================================================================
\section{Introduction}
% =============================================================================

Dieses Paper handelt von einem Projekt namens ''Uninvited Guests''\cite{uninvatedGuests} und der Analyse des Projektes mittels verschiedenen Denkweisen. Es werden folgende sechs Denkweisen in diesem Paper behandelt:
\begin{itemize}
	\item Computational Thinking
	\item Scientific Thinking
	\item Responsible \& Critical Thinking
	\item Design \& Creative Thinking
\end{itemize}

Anfangs werden die Denkweisen beschrieben. Anschließend wird das Video mit den jeweiligen Denkweisen analysiert. Es folgt eine Beschreibung der auftretenden Probleme und eine Lösung von diesen.\\
Am Ende des Papers erfolgt eine Diskussion, in welchen Punkten sich die verschiedenen Denkweisen überschneiden und/oder sich diese aber widersprechen. Zudem wird ein Ausblick gegeben, wie die genannten Gegenstände aus dem Projekt\cite{uninvatedGuests} konkret verbessert werden können und wie diese Verbesserung umgesetzt werden sollte.

\newpage
% =============================================================================
\section{Computational Thinking}
% =============================================================================

%!TEX root=../main.tex

\subsection{Was macht Computational Thinking aus?}
Computational Thinking beschäftigt sich mit der Bearbeitung komplexer Probleme, bei denen es ohne die Rechenleistung von Computern fast unmöglich wäre die gleichen Ergebnisse zu erzielen. Der Grund für die Komplexität kann sehr unterschiedlich sein, aber ein häufiger Grund sind große Datenmengen. Zur Lösung dieser Probleme werden Algorithmen herangezogen. Wichtige Konzepte die für dieses Problem relevant sind beinhalten:\cite{slidesComp}
\begin{itemize}
	\item Algorithmen (Spezifikation, Zerlegung, Abstraktion, Skalierbarkeit, Reduktion)\cite{slidesComp}
	\item Efficiency, correction, data represenation\cite{thinkingAboutCompThinking}
\end{itemize}

\subsection{Analyse}
Das System besteht aus mehreren Teilen, die gemeinsam Daten sammeln und analysieren um Aussagen über die Ernährung, Bewegung und den Schlaf einer Person zu treffen. \\

Die grundlegenden Probleme die das System behandeln muss sind folgende: 
\begin{itemize}
	\item Datenerhebung 
	\begin{itemize}
		\item Analyse des Essens durch Fork oder zählen der Schritte durch Cane 
	\end{itemize}

	\item Datentransport 
	\begin{itemize}
		\item Die Daten müssen zu den Computern transportiert werden in denen sie verarbeitet werden können (Annahme: Smart Objects machen das nicht selbst). 
		\item Weiteres müssen sie an die Kontakte verteilt werden können. 
	\end{itemize}  

	\item Analyse
	\begin{itemize}
		\item Es muss eine Analyse der Daten stattfinden um möglichst optimales Feedback liefern zu können. 
	\end{itemize}  

	\item User Interface 
	\begin{itemize}
			\item Die Ergebnisse müssen verschiedenen Benutzern präsentiert werden. 
	\end{itemize}
\end{itemize}

Hierbei werden verschiedene Algorithmen eingesetzt (in den geschwungenen Klammern ist die Zuordnung zu den Problemstellungen annotiert:
\begin{itemize}
	\item Zeitsteuerung  \{ \textit{User Interface }\}: 
	\begin{itemize}
		\item Das System wird aufgrund von Zeitintervallen gesteuert (Essenszeit, Schlafenszeit). Zu diesen Zeitpunkten müssen jeweils andere Algorithmen aktiviert werden.
	\end{itemize}

	\item Essensanalyse (Fork) \{ \textit{Datenerhebung} \}: 
	\begin{itemize}
		\item Die Gabel muss aufgrund der aufgespießten Nahrung deren Nährwert und Inhaltsstoffe erkennen können. Zudem müssen diese gegen eine Referenz verglichen werden. 
	\end{itemize}

	\item Drucksensor (Cane, Bed) \{ \textit{Datenerhebung} \} 
	\begin{itemize}
		\item Bed und Cane beinhalten jeweils einen Drucksensor der abgefragt werden muss ob die jeweiligen Tätigkeiten gerade verrichtet werden. 
	\end{itemize}

	\item Messaging \{ \textit{Datentransport, User Interface} \} 
	\begin{itemize}
		\item Die Kommunikation des Systems mit Benutzer\_innen erfolgt über Push Nachrichten. Im Notfall können auch andere Kontakte hinterlegt werden um die Verwendung der Smart Utensilien zu gewährleisten. 
	\end{itemize}

	\item Datenverteilung \{ \textit{Datentransport} \} 
	\begin{itemize}
		\item Die erfassten Daten können an die anderen Kontakte geschickt werden. 
	\end{itemize}
	
	\item Datenauswertung: \{ \textit{Analyse} \}
	\begin{itemize}
		\item Die Daten werden auf hier untersucht und mit Referenzwerten verglichen um passende Nachrichten an die Endgeräte versenden zu können. 
		\item Dieser Algorithmus ist sehr komplex (Annahme) und verwendet viele einfachere Algorithmen um dieses Ziel zu erreichen (suchen, sortieren, filtern, vergleichen \dots).
	\end{itemize}
\end{itemize}

Aus technischer Sicht konnte ich folgende Abstraktionen im System feststellen:
\begin{itemize}
	\item Smart Devices 
	\begin{itemize}
		\item Die Smart Devices sind die abstrakteste Schicht, sie verstecken die Komplexität des gesamten Systems hinter alltäglichen Gegenständen. 
	\end{itemize}

	\item Datenaustausch (Annahme)
	\begin{itemize}
		\item Der Datenaustausch erfolgt über verschiedene Protokolle und an verschiedenste Endgeräte (Server, Smart Devices, Smartphones von den Kontakten\dots ) und kapselt damit die Komplexität dieser Vorgänge ab
	\end{itemize}

	\item Sensoren
	\begin{itemize}
		\item Die I/O Schnittstellen der Sensoren kapseln deren interne Funktionalität ab. 
	\end{itemize}

	\item Datenverarbeitung
	\begin{itemize}
		\item Die Schnittstellen der Algorithmen (Input vs. Output) abstrahieren die Komplexität der zugrundeliegenden Operationen.
	\end{itemize}
\end{itemize}

\subsubsection{\textbf{Tagesanalyse}}
Ich habe die drei, im Video\cite{uninvatedGuests} betrachteten, Tage getrennt voneinander betrachtet und folgende Dinge festgestellt: \\
Am ersten Tag ist die Datenqualität optimal und das System kann wertvolles Feedback für den Mann liefern. Die einzelnen Geräte werden für die vorherbestimmten Tätigkeiten verwendet und treffen ihre Entscheidungen aufgrund valider Daten. Das Feedback für den Benutzer ist in diesem speziellen Fall negativ, aber das stimmt mit den gewonnenen Daten voll überein. \\
Am zweiten Tag wurden keine Daten an das System geliefert. Die zuständigen eingetragenen Kontakte wurden informiert (Annahme), aber es wurden von der Smart Fork und dem Smart Cane keine Daten gesammelt. Das Smart Bed konnte lediglich feststellen, dass die Person zu spät ins Bett kam, die Push Nachrichten wurden ignoriert. \\
Am dritten Tag wurden die gesammelten Daten stark verfälscht. Die Smart Fork wurde manipuliert in dem das Essen nicht wirklich verzehrt sondern nur daran gestochert wurde. Der Smart Cane wurde von einer anderen Person verwendet um auf die tägliche empfohlene Schrittanzahl zu kommen. Und der Drucksensor des Smart Bed wurde mit Hilfe von Büchern überlistet. All diese Manipulationen führten jeweils zu positivem Feedback, da die nötigen Parameter für die Algorithmen gegeben wurden.\\

\subsection{Problem Framing}
Das größte Problem aus der Sicht des Computational Thinking\cite{slidesComp} ist die Datenqualität, da die Analyse durch die Algorithmen mit falschem Input auch falschen Output liefert. Der Ansatz ist gescheitert, weil es Den Benutzer\_innen möglich ist die Daten zu fälschen. Wäre die Richtigkeit der Daten sichergestellt (könnte man das System also nicht ''überlisten'') könnten auch die Algorithmen im Hintergrund deutlich besseres Feedback liefern. \\
Ein weiteres Problem ist die Individualität bei den Aktivitäten. Die Smart Devices verwenden einen Standard und wenden diesen auf alle Personen an, wobei ein personalisiertes Ziel vermutlich zu besseren Ergebnissen führen würde. 

\subsection{Gestaltungsvorschlag}
Die Probleme könnten mit einer Kombination aus verschiedenen Ansätzen zumindest gemildert werden: 
\begin{enumerate}
	\item Authentifizierung: Um Fälschung der Daten durch andere zu vermeiden muss eine biometrische Authentifizierung stattfinden (zB Fingerabdruck oder Iris Scan) 
	\item Artificial Intelligence: Um verfälschte Daten besser zu erkennen könnte eine AI programmiert werden die die Daten auf historische Muster vergleicht und damit eine Angabe zur Relevanz der Daten machen kann.
	\item Verifikation: Sollten Daten stark von erwarteten Daten abweichen könnte eine zusätzliche Verifikation (z.B. per Foto) veranlasst werden. 
	\item Gamification\cite{gameDesignElements}: Um den Prozess der Datenerhebung für Benutzer\_innen attraktiver zu machen könnten diverse Methoden der Gamification verwendet werden. Zum Beispiel könnten Fortschritte an Applikationen wie Habitica\cite{habitica} gesendet werden.
\end{enumerate}

Verworfene Ansätze:
\begin{itemize}
	\item Interpolierende Algorithmen (Gesundheitsrisiko)
	\begin{itemize}
		\item Sind starke Abweichungen in den Daten vorhanden sein können diese mit Interpolation geglättet werden um etwaige Datenverfälschung zu mindern.
	\end{itemize}

	\item Bestrafen von Fehlverhalten (Ethik)
	\begin{itemize}
		\item Wird Fehlverhalten festgestellt könnten negative Auswirkungen dies für zukünftige Fälle unattraktiver machen.
		\item Beispiele hierfür sind strikte Pläne für die Ernährung oder Bewegung, oder direktere Methoden wie leichte Elektroschocks. 
	\end{itemize}

	\item Eingabe der Daten
	\begin{itemize}
		\item Die Eingabe der Daten funktioniert bereits optimal, sie ist in der Verwendung alltäglicher Gegenstände vollständig abstrahiert worden. 
	\end{itemize}
\end{itemize}

\newpage

% =============================================================================
\section{Scientific Thinking}
% =============================================================================

%!TEX root=../main.tex

\subsection{Was macht Scientific Thinking aus?}
Scientific Thinking\cite{slidesScientifc} beschäftigt sich mit dem Versuch eindeutige, gut begründete, geordnete und für gesichert erachtete Ergebnisse hervorzubringen. Das Ganze verfolgt das Ziel „unbestreitbares“ Wissen zu schaffen und darauf weiter aufbauen zu können. \\
Der Prozess dahinter läuft so ab, dass zuerst bestimmte Abläufe/Sachverhalte genau beobachtet werden, anschließend wird anhand dieser Beobachtungen eine Hypothese aufgestellt welche versucht die Beobachtungen zu erklären bzw. das beobachtete Problem zu lösen. Nachdem eine Hypothese aufgestellt wurde muss versucht werden diese oft genug zu überprüfen damit eindeutig gezeigt wird das die Hypothese allgemein gültig ist und nicht nur in einem überprüften Einzelfall gilt. Je nachdem ob die Hypothese der Überprüfung stand hält wird sie danach entweder als hervorgebrachtes Wissen betrachtet, immer weiter verbessert bis sie wissen hervorbringt oder wenn die Ergebnisse zu weit von den erwarteten entfernt ist auch einfach wieder verworfen.

\subsection{Analyse}
Grundlegend baut das System \cite{uninvatedGuests} schon mal auf Scientific Thinking auf. Das System kann nur dadurch funktionieren, dass irgendwann wissenschaftlich herausgefunden wurde wie lange Menschen täglich schlafen sollten, wie viel sie wovon essen sollten und wie viele Schritte man am Tag zumindest gehen sollte. Die Daten die erhoben werden müssen um zu überprüfen ob die „Anforderungen“ eingehalten werden entstehen bei der „Smart-Gabel“ durch Sensoren die den Nährwert der Speisen messen wenn man sie aufspießt, bei dem Bett durch Gewichtssensoren die Messen ob jemand/etwas im Bett liegt und beim „Smart-Gehstock“ wird durch Bewegungssensoren gemessen wie weit der Stock sich fortbewegt. Diese Daten werden an ein externes System übermittelt welches sie dann mit dem Optimum vergleicht und je nachdem ob dieses erreicht wurde positives oder negatives Feedback gibt. \\
Auf der anderen Seite könnte man auch das Verhalten des Mannes als Scientific Thinking betrachten.  \\
Am ersten Tag beobachtet er wie die Geräte darauf reagieren, wenn er sich verhält wie immer. Er lässt sich zwar zu ein paar extra Schritten motivieren gibt das Ganze aber dann relativ schnell wieder auf. Da das Feedback des Systems am ersten Tag sehr negativ ausfällt stellt er die Hypothese auf, dass es doch wesentlich angenehmer wäre, wenn diese Geräte ihn nicht ständig daran erinnern würden was er Essen sollte, wie viel er sich bewegen sollte und wann er schlafen sollte.  \\
Daraufhin entschließt er sich diese Hypothese am nächsten Tag zu überprüfen indem er die Geräte neben sich liegen lässt beziehungsweise sie so verstaut das diese ihm nicht mehr durch ständige Erinnerungen auf die Nerven geht. Dies gelingt jedoch leider nicht da ihm seine Kinder daraufhin auf die Tatsache ansprechen, dass die Geräte nicht verwendet werden. Das war natürlich nicht sein gewünschtes Ergebnis da er obwohl die Geräte nicht benutzt wurden trotzdem auf sein „Fehlverhalten“ hingewiesen wurde.  \\
Darum entschließt er sich seine Hypothese zu verbessern und die Geräte am dritten Tag nicht mehr nicht zu verwenden, sondern sie auszutricksen. Das macht er indem er mit seiner „Smart-Gabel“ nicht wirklich isst, sondern damit nur in gesundem Essen herumstochert, seinen „Smart-Gehstock“ nicht selbst benutzt, sondern ihn jemanden anderen gibt um die Bewegung für ihn zu machen und indem er die Gewichtssensoren des Bettes hinters Licht führt indem er es mit Büchern beschwert. 
Nachdem das Feedback welches er zu diesem Verhalten erhält ausschließlich positiv ist hat er durch beobachten, anschließende Hypothesen aufstellen/überprüfen und daraus Schlussfolgerungen ziehen das Wissen erlangt wie er mit diesen Geräten leben kann ohne das sie ihm auf die Nerven gehen.  

\subsection{Problem Framing}
Das Hauptproblem des Ansatzes dieser Geräte ist, dass Angenommen wird das Menschen, wenn sie darauf hingewiesen werden das andere Verhaltensmuster wissenschaftlich gesünder für sie wären, sie ihr Verhalten dementsprechend anpassen würden. Jedoch ist es zum einen nun mal so, dass Menschen nicht immer das tun was für ihre physische Gesundheit am besten ist, wenn man dafür seine persönlichen Wünsche vernachlässigen muss. \\
Ein weiteres Problem besteht darin, dass diese Daten nicht zwingender maßen vertrauenswürdig sind, sie aber das einzige Kriterium für den Erfolg der Technologie sind. Wie man bei dem Mann in dem Video schon gesehen hat sind diese relativ leicht zu verfälschen, wenn man sich nicht an die „Anforderungen“ halten möchte und damit ist der Sinn des Systems sehr schnell übergangen und dadurch auch zur Gänze nutzlos. 

\subsection{Gestaltungsvorschlag}
Um diese beiden Probleme beheben zu können müsste man dafür sorgen, dass das System Benutzerfreundlicher wird und dafür gesorgt wird das der Benutzer die gesetzten Ziele auch wirklich selbst einhalten will.  
Dies wäre am einfachsten möglich indem man bei der Einrichtung des Systems genauer auf die Wünsche und Gewohnheiten des Nutzers eingeht. Dafür wäre zum Beispiel erforderlich, dass das System nicht ab der ersten Benutzung „Anforderungen“ stellt die dem Optimum entsprechen, sondern zunächst einmal für einen gewissen Zeitraum (ca. eine Woche) nur misst wie das Ess-, Schlaf- und Bewegungsverhalten der Person normalerweise ist. \\
Anschießend können je nachdem unterschiedliche „Anforderungen“ gestellt werden die sich den persönlichen Gewohnheiten anpassen und sich dann mit der Zeit Schritt für Schritt immer weiter ans Optimum annähert. Dadurch würden wesentlich kleinere Schritte schon ein positives Feedback auslösen und das würde sehr zur Motivation beitragen die Geräte langfristig zu benutzen und sich Schritt für Schritt ans Optimum anzunähern. 
\newpage

% =============================================================================
\section{Responsible \& Critical Thinking}
% =============================================================================

%!TEX root=../main.tex

\subsection{Beschreibung der Denkweise}
\subsubsection{\textbf{Responsible Thinking}}  
Beim Responsible Thinking\cite{slidesResp} ist es wichtig, sich über die Auswirkungen des Systems auf die Gesellschaft und die Umwelt Gedanken zu machen. Durch die Verwendung/Entwicklung von Produkten hat man auch eine gewisse Verantwortung. Das Responsible Thinking geht über die gesetzlichen Anforderungen hinaus und befasst sich auch mit ethischen Fragen, wobei sowohl die eigenen als auch die gesellschaftlichen Werte hinterfragt werden. Aktuelle und wichtige Punkte bei den ethischen Fragen sind die Datensicherheit und die Überwachung. 
\\
\subsubsection{\textbf{Critical Thinking}}  
Beim Critical Thinking\cite{slidesCritical} sollen alle Seiten eines Problems beleuchtet werden, um so Voreingenommenheit zu beseitigen und Stereotypen zu entkräften. Dabei ist es wichtig, auch auf Kritikpunkte einzugehen, die nicht der eigenen Meinung entsprechen. Wenn eine Innovation vom Großteil der Gesellschaft gutgeheißen wird, muss dies nicht bedeuten, dass sie auch für eine Minderheit tragbar ist. Beim vorliegenden Problem sind vor allem die Überwachung und, aufgrund des Systemdesigns, eine mögliche Stigmatisierung kritisch zu betrachten. 

\subsection{Analyse der im Video „dokumentierten“ Nutzung des Systems}
Für das Responsible Thinking ist es wichtig, die ethischen und moralischen Aspekte einer solchen Erfindung zu beleuchten. 
Ist es moralisch korrekt, einen Menschen ständig zu überwachen? Sollte ein Mensch nicht die Freiheit haben, tun und lassen zu können, was er/sie möchte? Sollte man nicht selbst entscheiden können, was man wann isst, welche Hobbies man pflegt bzw. wann man schlafen geht? \\
Eine ständige Überwachung mindert die Lebensqualität des Menschen. Diese Feststellung muss nun mithilfe des Critical Thinking überprüft werden. Aus Sicht der betroffenen Familienmitglieder ist das System sinnvoll und praktisch. Sie haben einen direkten Einblick in den Tagesablauf der „überwachten“ Person. Essverhalten, Bewegung und Schlafenszeit können gemessen und überprüft werden. Dies alles kann aus Sorge um die/den Angehörige(n) und ohne Hintergedanken geschehen, aber auch um das eigene Gewissen zu beruhigen, da man keine Zeit hat, sich selbst um diese Person zu kümmern. \\
Für die „überwachte“ Person ist diese Überwachung jedoch einengend und in gewisser Weise menschenunwürdig. Sie wird in ihren Freiheiten eingeschränkt und bevormundet. Sie darf nicht selbst entscheiden, was sie isst, wann sie Bewegung machen will oder auch nicht, ja selbst wann sie schlafen gehen soll, wird vorgegeben. Das hier gezeigte System verstärkt noch die vorherrschende Meinung in unserer Gesellschaft, dass ältere Menschen unselbständig sind und Hilfe benötigen. \\
Ein ebenso wichtiger Punkt ist eine durch dieses System möglicherweise auftretende Stigmatisierung älterer Leute. Durch das Design des Systems im Video kann es sein, dass das System von vielen Personen der Zielgruppe abgelehnt wird. Das Design lässt darauf schließen, dass man aus der Norm fällt, also z.B. hilfsbedürftig erscheint. Gabel und Gehstock sind aufgrund des Designs nicht unbedingt für außerhalb der eigenen Wohnung geeignet. Vor allem die Farbe dieser Gegenstände ist sehr auffällig. In der Öffentlichkeit würde man damit Aufsehen erregen bzw. auf jeden Fall auffallen. Man könnte daher z.B. als „alt und gebrechlich“ eingestuft werden. Daher könnte dieses System in der Öffentlichkeit zu einer Stigmatisierung der Zielperson führen.

\subsection{Tagesanalyse}
\subsubsection{\textbf{Tag 1}}
Das smarte System (Gabel, Gehstock, Bett) gibt dem Mann (der Zielperson) die entsprechenden Anweisungen. Die Gabel zeigt an, wieviel Salz, Fett und Kalorien in der Mahlzeit enthalten sind bzw. wie oft am Tag gegessen werden soll. Der Gehstock gibt die Anzahl der Schritte pro Tag vor und das Bett zeigt an, wann es Schlafenszeit ist. Der gesamte Tagesrhythmus wird durch das smarte System vorgegeben. Für selbständige Entscheidungen bzgl. Essen, Spaziergang und Schlafenszeit ist kein Spielraum gegeben.
\subsubsection{\textbf{Tag 2}}
Auch an diesem Tag gibt das smarte System die notwendigen Anweisungen. Diese werden jedoch von der Zielperson ignoriert. Prompt erfolgt eine Rückmeldung der Kinder, ob alles in Ordnung sei bzw. warum das smarte System nicht benutzt wird. \\
Die Kinder sorgen sich auf der einen Seite um das Wohl der Zielperson, daher haben sie ja auch für diese das smarte System besorgt. Auf der anderen Seite beschränkt sich der Kontakt zur Zielperson jedoch nur darauf, zu eruieren, warum das System nicht benutzt wird. Der direkte Kontakt zwischen den Kindern und der Zielperson ist nicht gegeben. Die Kinder telefonieren nicht, sondern schicken eine SMS. Dies kann das Gefühl von Alleinsein noch verstärken, da man keinen Ansprechpartner für seine Probleme hat. Die äußere Kontrolle schränkt die Selbstentscheidung der Zielperson stark ein. Sie fühlt sich bevormundet und nicht als selbstbestimmte Person.
\subsubsection{\textbf{Tag 3}}
Die genervte Zielperson überlistet das System und erntet von diesem „dafür“ Lob. Die Zielperson stochert mit der Gabel im gesunden Essen, lässt jemand anderen die vorgegebene Schrittanzahl tun und gaukelt dem Bett vor, darin zu liegen und zu schlafen. \\
Die Zielperson scheint sich sichtlich wohler zu fühlen. Außer der sehr positiven Rückmeldung des Systems wird sie auch nicht von den Kindern wegen Fehlverhaltens oder Nichtbenutzung des Systems überwacht und „kritisiert“. 

\subsection{Problem Framing}
Im Video\cite{uninvatedGuests} erkennt man deutlich, dass das smarte System starken Einfluss auf den persönlichen Tagesablauf und die Privatsphäre der Person hat. Egal ob beim Essen, beim Lesen oder Fernsehen, immer wird die Person vom System darauf aufmerksam gemacht, dass etwas anderes zu tun sei. Verwendet sie das smarte System nicht, wird sie von den Familienmitgliedern kontaktiert, ob ohnehin alles in Ordnung sei. Die Haltung der Kinder spiegelt die Haltung der Gesellschaft gut wider, dass ältere Menschen nicht für sich selbst sorgen bzw. auf ihre Gesundheit achten können und daher ständig kontrolliert und „bemuttert“ werden müssen.  \\
Ein großes Problem stellen auch die hochgesteckten Ziele des Systems dar. Die Zielperson hat sichtlich Probleme mit den Füßen und daher tut sie sich beim Gehen sichtlich schwer. Daher würde die Person den Tagesablauf viel lieber selbst planen und den Tag gemütlich verbringen. Das smarte System ist in diesem Fall ein Stressfaktor für die Zielperson. \\
Die Nutzung der durch das smarte System gespeicherten Daten stellt ebenso ein großes Problem dar. So könnten diese Daten von der Herstellungsfirma an Dienstleister wie Versicherungen weitergegeben werden. Dieser Datenmissbrauch könnte z.B. zu unterschiedlich hohen Versicherungsprämien führen, was wiederum zu einer Diskriminierung jener Personen führen könnte, die das smarte System nutzen.  

\subsection{Gestaltungsprozess}
Um zu einem besseren Entwurf zu kommen, müsste das smarte System mehr an die jeweilige Zielperson angepasst werden. Eine schrittweise Umstellung von den derzeitigen Lebensgewohnheiten auf eine gesündere Lebensweise wäre empfehlenswert. Auch sollte das System an die jeweiligen gesundheitlichen Gegebenheiten der Zielperson adaptiert werden können. \\
Wichtig wäre daher bei der Implementierung eines solchen Systems die Mitwirkung eines Arztes oder einer Ärztin des Vertrauens. Dadurch könnten der Nutzen und die Ziele dieser Systeme besser an die Zielpersonen vermittelt werden und gemeinsam Pläne zur Umsetzung durch schrittweise Anpassung an die Lebensgewohnheiten gemacht werden. \\
Um Missbrauch der Daten und eine Überwachung zu verhindern, sollte es die Möglichkeit geben, diese „Überwachung durch Dritte“ zu deaktivieren. Dies könnte durch Speicherung der Daten am jeweiligen Gerät erfolgen, sodass Dritte darauf keinen Zugriff haben oder nur mit Einverständnis der jeweiligen Zielperson. \\
Als Gesellschaft sind wir gefordert umzudenken. Ältere Menschen mögen gebrechlich sein, dies gibt uns jedoch nicht das Recht, sie zu entmündigen und ihnen vorzuschreiben, was sie zu tun haben. Jede(r), auch ältere Personen, hat das Recht, sein Leben, seinen Tagesablauf so zu gestalten wie sie/er es will. Das Leben lebenswert erhalten, auch im Alter, sollte das Ziel auch von smarten Systemen sein. Wichtig ist die Einbindung solcher Systeme in den Lebensalltag, jedoch sollen sie nicht das Alltagsleben kontrollieren. 
\newpage

% =============================================================================
\section{Design \& Creative Thinking}
% =============================================================================

%!TEX root=../main.tex

\subsection{Beschreibung Denkweise}
\subsubsection{\textbf{Design Thinking}}
Design Thinking und Human Centered\cite{slidesDesign} beschreibt den Design Prozess und wie das Design auf Nutzer und Nutzerinnen wirkt. Im Falle der Smart Devices beispielsweise geht es darum die Geräte so zu gestalten, dass zusätzliche ''smarte'' Funktionen bestehen bleiben, ohne das die Hauptfunktionalität verloren geht/darunter leidet. Bei der Gabel zum Beispiel soll der Nutzer/die Nutzerin damit noch angenehm essen können. \\
Das Hauptproblem liegt darin die Hauptfunktionalität beizubehalten, zusätzliche Funktionen einzubauen und dem Nutzer/der Nutzerin Feedback zu geben, ohne das dieser/diese sich genervt oder genötigt fühlen. Im Falle von Thomas\cite{uninvatedGuests} ist dies nicht ganz der Fall, da er beim nicht Benutzen der Gegenstände permanent Benachrichtigungen und Erinnerungen bekommt.
\subsubsection{\textbf{Creative Thinking}}
Creative Thinking\cite{slidesCreative}
\subsection{Analyse des Videos}
\subsubsection{\textbf{Generelle Dokumentation zum Design}} 
Bei dem Design der verschiedenen Smart Devices wurde hauptsächlich das Ziel verfolgt die Geräte nicht zu sehr zu verändern. Die Geräte sollen in der Bedienung gleich wie die Vorgängerobjekte sein. Das Ziel ist es ja nicht die Nutzung zu optimieren, sondern durch gesammelte Daten eine Person gesünder zu machen oder diese Daten zu merken. \\
Die Smart Devices sind im Video hervorgehoben, da diese die Hauptrolle im Video darstellen. Die gewählte Farbe Gelb hebt die Gegenstände deutlich hervor und die Gegenstände wirken abstrakt verglichen mit den anderen Gegenständen. Zudem sind manche Smart Devices abstrakter im Aufbau gestaltet. Die Gabel ist beispielsweise dicker als eine normale Gabel, dies wurde wahrscheinlich aufgrund der vorhanden und eingebauten Elektronik gemacht. \\
\subsubsection{\textbf{Tag 1}} 
Am ersten Tag von Thomas alltäglichen Leben wird die Funktionsweise der Smart Devices vorgestellt. \\
Die Gabel soll Thomas dabei helfen seine Ernährung gesünder zu gestalten, indem das gegessene Essen mittels Sensoren überprüft wird. \\
Der Gehstock ist eine Art Schriftzähler, welcher überprüft, ob Thomas auch genug Bewegung pro Tag macht. \\
Die Matratze erinnert Thomas, wann er am besten Schlafen gehen soll. \\
Alle Smart Devices sind mit dem Handy verbunden um dem Nutzer/der Nutzerin Benachrichtigungen über den aktuellen Stand zu schicken, beziehungsweise Erinnerungen, falls Thomas etwas vergisst.
\subsubsection{\textbf{Tag 2}} 
Thomas ignoriert die Smart Devices und benutzt diese nicht. Zuerst wird Thomas mit Erinnerungen von den Smart Devices zugeschüttet, dass er die Smart Devices benutzen soll. Diese Erinnerungen ignoriert er auch, jedoch benachrichtigen die Smart Devices auch seine Kinder und diese melden sich dann. Thomas Kinder überwachen Thomas also und informieren sich über seine Gesundheit mittels den Smart Devices.
\subsubsection{\textbf{Tag 3}} 
Thomas trickst die Smart Devices aus und verschafft sich so Ruhe von den Erinnerungen und auch seinen Kindern, welche Ihn überwachen. Auf den ersten Blick wirkt es als würde Thomas seine Gesundheit (Er kann problemlos ungesund essen, kann den ganzen Tag lesen/fernsehen und hat einen unregulierten Schlafrhythmus) negativ beeinflussen. Jedoch hat das Austricksen auch positive Seiten. Er kann wie gewohnt sein Leben leben, ohne das er von Smart Devices und seinen Kindern überwacht wird und seine Kinder machen sich weniger Sorgen um ihn, da sie sehen, dass alles passt. \\
Generell ist der Designvorschlag mit der Überwachung nicht gut. Ein Mensch sollte sein Leben frei leben können. Vor allem das die Kinder durch gewisse Apps seine Gesundheit überwachen und nicht persönlich Vorbeischauen finde ich ein wenig unmenschlich. Wenn sie sich wirklich Sorgen um Thomas machen, dann könnten sie ihn auch von Zeit zu Zeit besuchen. \\
\subsection{Problem Framing}
Das Smart Devices wird nicht direkt für den tatsächlichen Nutzer/der tatsächlichen Nutzerin designt, sondern für die Personen, welche es kaufen für die Überwachung vom gesundheitlichen Zustand des tatsächlichen Nutzers/der tatsächlichen Nutzerin. In Thomas Fall wurde das Smart Device also nicht für ihn, sondern für seine Kinder designt, welche ihn überwachen und das ist auch schon das grundlegende Problem. Thomas ist der Nutzer von den Objekten und er sollte diese auch selbst steuern können. Es wird zwar nicht erwähnt, wer Zugriff auf die Einstellungen (z.B. Bettzeit) von den Smart Devices hat, jedoch denke ich das die Kinder die Administrationsrechte haben, da sie auch die Daten einsehen können. Somit hat Thomas weniger Freiheit über sein Leben.
\subsection{Gestaltungsvorschlag}
An sich sind die Smart Devices nicht schlecht designt und auch von ihrer Funktion nützlich. Jedoch finde ich sollte Thomas die vollständigen Administrationsrechte für die Smart Devices bekommen und er sollte auch entscheiden dürfen, ob seine Kinder Einsicht auf die gesammelten Daten von den Geräten bekommen sollten. \\
Zudem sollten die Smart Devices eine Art Pause Modus haben, in welchem Thomas nicht unbedingt auf alles achten muss ohne gleich mit Benachrichtigungen voll geschüttet zu werden. Selten etwas ungesundes zu Essen, einen faulen Couchtag zu haben und auch die regulierte Schlafzeit zu ignorieren wirkt sich nicht negativ auf die Gesundheit aus. 

\newpage

% =============================================================================
\section{Diskussion}
% =============================================================================

Welche Querverbindungen, Gemeinsamkeiten, Spannungsfelder, Widersprüche, Konflikte etc. sehen sie zwischen den unterschiedlichen Herangehensweisen?

Wie würden sie dieser Verbindungen bewerten oder priorisieren?

Wie könnten diese Denkweisen ineinander greifen um die Probleme in ``Uninvited Guest'' zu lösen?

% =============================================================================
\section{Ausblick}
% =============================================================================

Wenn sie damit beauftragt wären die Technologien in `Uninvited Guest'' zu verbessern, wie würden sie das konkret angehen?

Welche Studien würden sie machen, was würden sie entwickeln, was würden sie wie testen?




\bibliography{Literature}
\bibliographystyle{ACM-Reference-Format}

\end{document}
